\documentclass{beamer}

% packages
\usepackage {lmodern}
\usepackage {graphicx}
\usepackage {caption}
\usepackage {gensymb}
\usepackage {colorspace}
\usepackage[LGRgreek]{mathastext}
\usepackage[T1]{fontenc}
\setbeamerfont{footnote}{size=\tiny}
\usepackage[backend=biber, autocite=superscript]{biblatex}
\usepackage {listings}

\lstdefinestyle{customc}{
  belowcaptionskip=1\baselineskip,
  breaklines=true,
  frame=L,
  xleftmargin=\parindent,
  language=bash,
  showstringspaces=false,
  basicstyle=\footnotesize,
  keywordstyle=\bfseries\color{green!40!black},
  commentstyle=\itshape\color{purple!40!black},
  identifierstyle=\color{blue},
  stringstyle=\color{orange},
}

\lstdefinestyle{customasm}{
  belowcaptionskip=1\baselineskip,
  frame=L,
  xleftmargin=\parindent,
  language=bash,
  basicstyle=\footnotesize,
  commentstyle=\itshape\color{purple!40!black},
}

\lstset{escapechar=@,style=customc}


\usetheme{Hannover}

%\definecolor{optumOrange}{cmyk}{0, .62, .95, 0}
%\setbeamercolor{structure}{fg=optumOrange}


\graphicspath {{images/}}


\AtBeginBibliography{\scriptsize}
\bibliography{oralDefense.bib}

%\setbeamertemplate{bibliography item}{}

\title{Using Vim}
\author{Justin Chao}
\date{November 6, 2017}
\titlegraphic{\includegraphics[width=0.4\textwidth, height=0.2\textwidth]{vim}}

\begin{document}
\frame{\titlepage}

\section{Introduction}

\begin{frame} {Introduction} {The Approach}
    \begin{itemize}
        \item Introduction \\ 
        \item Basics \\
        \item Getting Stuff Done \\
        \item Advanced
    \end{itemize}
\end{frame}

\subsection{What is Vim}
\begin{frame} {Introduction} {What is Vim}
    Vim is a highly configurable, \underline{modal} text editor built to make creating and changing any kind of text
    very efficient. \\

    \begin{itemize}
        \item System administration\\
        \item Programming\\
        \item Working with HTML, LaTeX, or other markup languages\\
        \item Heavy editing of plain text files\\
    \end{itemize}
\end{frame}

\subsection{Learning Vim}
\begin{frame}[t]{Learning Vim}
    \centering
    \includegraphics[width=0.8\textwidth]{learningVim}
\end{frame}

\subsection{Why Vim}
\begin{frame} {Introduction} {Why Vim}
    \begin{itemize}
        \item It's ubiquitous. \\It's installed on every Unix-like system, and it's free!\\
        \item It's scalable. \\You can use it just to edit config files or it can become your entire writing platform. \\
        \item It's powerful. \\Because it works like a language vim takes you from frustrated to demigod very quickly. \\
    \end{itemize}
\end{frame}


\begin{frame} {Introduction} {Levels to Vim Mastery}
    \begin{itemize}
        \item Level 0: knows nothing about vim \\
        \item Level 1: knows vim basics \\
        \item Level 2: knows visual mode \\
        \item Level 3: knows various motions \\
        \item Level 4: not needing visual mode \\
        \item Level 5: magic things are happening on the screen \\
    \end{itemize}
\end{frame}


\section{Vim as a Language}

\begin{frame} {Vim as a Language}
\begin{table}
    \centering
    \label{tab:label}
    \begin{tabular}{l|c|l}
        Verbs    & d   & delete \\
                 & c   & change           \\
                 & y   & yank (copy)      \\
                 & v   & visually select  \\
        \hline
        Modifers & i   & inside \\
                 & a   & around \\
                 & NUM & number (1,2,10) \\
                 & t   & searches for something and stops before it \\
                 & f   & searches for that thing and lands on it \\
        \hline
        Nouns    & w   & word \\
                 & s   & sentence \\
                 & p   & paragraph \\
                 & t   & tag (think HTML/XML) \\
                 & b   & block (think programming) \\
    \end{tabular}
\end{table}
\end{frame}


\subsection{Building Sentences}
\begin{frame}[fragile] {Vim as a Language} {Building Sentences (Commands)}
    \centering
    Verb - Modifier - Noun

    \begin{lstlisting}[language=sh]
    # delete two words
    d2w

    # change inside sentence
    cis

    # yank (copy) inside paragraph 
    yip

    # change to the < character
    ct<
    \end{lstlisting}
\end{frame}


\section{Getting Things Done}

\subsection{Working With Your File}
\begin{frame}[t]{Getting Things Done} {Working With Your File}
    \begin{table}[htpb]
        \centering
        \begin{tabular}{r|l}
            vi/vim file & open your file in vim \\
            :w & write your changes to the file \\
            :q! & quit vim without saving your changes \\
            :wq & write your changes, and quit \\
        \end{tabular}
    \end{table}
\end{frame}


\subsection{Moving Around in Your Text}
\begin{frame}[fragile]{Getting Things Done} {Moving Around in Your Text} 
    \begin{lstlisting}
    # search for include
    /include<CR>

    # jump forward and land on the < character
    f<

    # jump forward and land right before the < character
    t<

    # change to and including the < character
    t<
    \end{lstlisting}  
\end{frame}

\subsection{Searching Your Text}
\begin{frame}[fragile]{Getting Things Done} {Searching Your Text} 
    \begin{lstlisting}
    # search for include
    /include<CR>

    # jump forward and land on the < character
    f<

    # jump forward and land right before the < character
    t<

    # change to and including the < character
    ct<
    \end{lstlisting}  
\end{frame}





\subsection{Basic Motions}
\begin{frame}[t]{Getting Things Done}
    
\end{frame}





\begin{frame}
\frametitle{References}
\printbibliography
\end{frame}
\end{document}
